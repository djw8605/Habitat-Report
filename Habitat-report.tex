\documentclass[12pt, notitlepage]{article}

\usepackage{fullpage}
\usepackage{listings}
\usepackage{color}

\definecolor{gray}{rgb}{0.5,0.5,0.5}
\definecolor{dkgreen}{rgb}{0,0.6,0}
\definecolor{mauve}{rgb}{0.58,0,0.82}

\lstset{
backgroundcolor=\color{white},
frame=single,
title=\lstname, 
keywordstyle=\color{blue}, 
commentstyle=\color{dkgreen}, 
stringstyle=\color{mauve}
}

\begin{document}



\title{Habitat Simulation}
\author{Abel Souza, Derek Weitzel, Douglas Silva}
\date{December 7, 2012}

\maketitle

\section{Introduction}

\section{Implementation}

\section{Evaluation}

In order to evaluate the habitat parallelization, we created input parameters that would span possible parameters used by researchers in the simulation.  The application has two main parameters, the landscape size and the number of starting agents.  Since we parallelized the execution of agents, we expect that the application will scale well with the number of agents.  Also, a larger landscape can handle more agents.  We created input parameter files that included different landscape dimensions and starting agents, as shown in Table \ref{tab:parameters}.

\begin{table}[ht]
\centering
\begin{tabular}{ c | c }
\textbf{Dimensions} & \textbf{Starting Agents} \\ 
\hline \hline
129 x 129 & 300 \\
129 x 129 & 800 \\
200 x 200 & 5000 \\
500 x 500 & 800 \\
500 x 500 & 50000 \\
1024 x 1024 & 1000
\end{tabular}
\caption{Input parameters} \label{tab:parameters}
\end{table}

The habitat executable opens the files \texttt{habitat.in} and \texttt{makeland.in} when it begins execution.  \texttt{habitat.in} contains both the dimensions of the landscape and the number of starting agents.  \texttt{makeland.in} only includes the dimensions of the landscape.

Evaluation jobs where submitted to Tusker with the submission script shown in Listing \ref{lst:submissionfile}.

\begin{figure}[ht]
\lstinputlisting[language=bash,title=pbs.sh,caption={Submission file used for evaluation},label={lst:submissionfile}]{Include/pbs.sh}
\end{figure}

As you can see from the submit script in Listing \ref{lst:submissionfile}, the habitat runs execute in isolation.  First it creates a temporary directory and copies the input files and habitat executable into the temporary directory.  Next it starts the execution and times it.  The stderr of the job includes the job run time.  For the purposes of timing, we use the \texttt{real} time output.

The parameters in Table \ref{tab:parameters} where executed at 7 different core counts: 1, 2, 4, 8, 16, 32, 48.  

\section{Conclusions}

\end{document}